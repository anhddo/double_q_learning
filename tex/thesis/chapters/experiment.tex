The Univariate Fourier Series
The Fourier series is used to approximate a periodic function; a function $f$ is periodic with period $T$ if $f(x+T)=$ $f(x), \forall x .$ The $n$ th degree Fourier expansion of $f$ is:
$$
\bar{f}(x)=\frac{a_{0}}{2}+\sum_{k=1}^{n}\left[a_{k} \cos \left(k \frac{2 \pi}{T} x\right)+b_{k} \sin \left(k \frac{2 \pi}{T} x\right)\right]
$$
with $a_{k}=\frac{2}{T} \int_{0}^{T} f(x) \cos \left(\frac{2 \pi k x}{T}\right) d x$ and with $b_{k}=$
$\frac{2}{T} \int_{0}^{T} f(x) \sin \left(\frac{2 \pi k x}{T}\right) d x .$ For the remainder of this paper
we assume for simplicity that $T=2,$ with the variables of the function we wish to approximate scaled appropriately. In the RL setting $f$ is unknown so we cannot compute $a_{0}, \ldots, a_{n}$ and $b_{1}, \ldots, b_{n},$ but we can instead treat them as parameters in a linear function approximation scheme, with:
$$
\phi_{i}(x)=\left\{\begin{array}{ll}
1 & i=0 \\
\cos \left(\frac{i+1}{2} \pi x\right) & i>0, i \text { odd } \\
\sin \left(\frac{i}{2} \pi x\right) & i>0, i \text { even }
\end{array}\right.
$$

The Multivariate Fourier Series
The $n$ th order Fourier expansion of the multivariate function $F(\mathbf{x})$ with period $T$ in $d$ dimensions is:
$$
\bar{F}(\mathbf{x})=\sum_{\mathbf{c}}\left[a_{\mathbf{c}} \cos \left(\frac{2 \pi}{T} \mathbf{c} \cdot \mathbf{x}\right)+b_{\mathbf{c}} \sin \left(\frac{2 \pi}{T} \mathbf{c} \cdot \mathbf{x}\right)\right]
$$

where $\mathbf{c}=\left[c_{1}, \ldots, c_{d}\right], c_{j} \in[0, \ldots, n], 1 \leq j \leq d .$ This
results in $2(n+1)^{d}$ basis functions for an $n$ th order full Fourier approximation to a value function in $d$ dimensions, which can be reduced to $(n+1)^{d}$ if we drop either the sin or cos terms for each variable as described above. We thus define the $n$ th order Fourier basis for $d$ variables:
$$
\phi_{i}(\mathbf{x})=\cos \left(\pi \mathbf{c}^{i} \cdot \mathbf{x}\right)
$$
where $\mathbf{c}^{i}=\left[c_{1}, \ldots, c_{d}\right], c_{j} \in[0, \ldots, n], 1 \leq j \leq d .$ 

In this chapter, we empirically evaluate the proposed optimistic VI algorithm, and compare it to
standard VI, as well as VI with greedy exploration. The goal of our experiments is to provide insight
into the benefits of using optimism in value iteration. We initialize all value functions optimistically.
We evaluate algorithms on the following environments.\\
 \textbf{N-chain}
The N-chain environment involves a long chain of N states with two actions: left and right.
The agent can only move right in state 1 and move left when it is in state N. The observed state is
the current state in the chain (i.e from 1 to N). Initially, agent starts at state 1. In state 1, the agent
receives reward 0.01 deterministically. In state N, the agent receives reward 1 with probability 0.1
and zero otherwise. The reward is zero in all other states. The optimal policy is to head right at every
time step and keep collecting reward at state N. The N-chain environment is illustrated in Figure 1.\\
 \textbf{CartPole}
 (Barto et al., 1983). In the CartPole environment, a pole is attached by an unactuated joint
to a cart, which moves along a frictionless rail. The system is controlled by pushing the cart to the
left or right. The observation consists of the position and velocity of the cart, pole angle, and pole
velocity at the tip. The pole starts upright, and the goal is to prevent it fall down. The reward is +1
for every time step that the pole remains upright. The episode ends if the pole angle is more than 12
degrees from vertical, or the cart moves more than 2.4 units from the center, or after 500 steps.\\
 \textbf{Mountain Car}
 (Moore, 1990). In the Mountain Car environment, a car is in a position between two
mountains. The goal is to drive the car up to the mountain on the right. However, the car needs to
drive back and forth to build up momentum because the car's engine is not strong enough to drive
up the mountain in a single pass. The agent receives a reward -1 at each time step or receives zero
reward if it successfully drive up the mountain. Each episode ends after 200 time steps or when the
car reaches the goal.\\
 \textbf{Acrobot }
(Geramifard et al., 2015) The Acrobot is a system with two links pendulum, where only the
second joint between the two links is actuated. Initially, the links are hanging downwards, and the
goal is to swing the end of the lower link up to a given height. The environment has six observations,
consists of the sine and cosine of the two rotational joint angles and the joint angular velocities. The
action is either applying +1, 0 or -1 torque on the joint between the two pendulum links. The reward
is -1 at each time step until the episode ends. The episode terminates if the agent successfully swings
up the pendulums, or after 500 time steps.
For CartPole, Mountain Car and Acrobot environments, we construct multivariate Fourier basis
features (Konidaris et al., 2011) of order 2 from the given states and actions. We set action features to
be binary action indicator vectors, and set $ \phi (s, a)$ to be the cross-product between state and action
features. For  $\varepsilon$-greedy algorithm, we choose  $\varepsilon $ = 0.1 where, with probability ", the agent selects
an action uniformly at random. Each algorithm runs for 25000 time steps and we keep track of the
total rewards at each episode. For each algorithm and environment we display the mean and standard
deviation of the reward per episode over 50 runs.
The results are shown in Figures 2. It can be seen that on N-chain, optimistic VI drastically outper-
forms both standard value iteration and the version with "-greedy exploration. On the experiments
with function approximation, optimistic VI consistently performs well on CartPole, Mountain Car
and Acrobot environments.

\begin{figure}[ht!]
    \centering
    \begin{subfigure}[b]{0.47\textwidth}
        \includegraphics[width=\textwidth]{image/n_chain}
        \caption{Evaluation on N-chain environment.}
        \label{fig:nchain}
    \end{subfigure}
    \begin{subfigure}[b]{0.47\textwidth}
        \includegraphics[width=\textwidth]{image/pole1}
        \caption{Evaluation on CartPole environment.}
        \label{fig:pole}
    \end{subfigure}
    \hfill
    \begin{subfigure}[b]{0.47\textwidth}
        \includegraphics[width=\textwidth]{image/car}
        \caption{Evaluation on Mountain Car environment.}
        \label{fig:car}
    \end{subfigure}
    \begin{subfigure}[b]{0.47\textwidth}
        \includegraphics[width=\textwidth]{image/acrobot}
        \caption{Evaluation on Acrobot environment.}
        \label{fig:acrobot}
    \end{subfigure}
    \caption{Experimental results with mean and standard deviation of 50 runs.}
    \label{fig:experiment}
\end{figure}

