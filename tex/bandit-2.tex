\documentclass{beamer} 
\usepackage{amsmath}
\usepackage{amsfonts}
\usetheme{madrid}
\author{Anh Do}
\institute{}
\date{}
\title{Chapter 2: Foundation of probability}

\newcommand{\f}{\mathcal{F}}
\newcommand{\g}{\mathcal{G}}
\newcommand{\X}{\mathcal{X}}
\newcommand{\x}{\mathbf{x}}
\DeclareMathOperator*{\argmax}{argmax} 
\DeclareMathOperator*{\argmin}{argmin} 

\begin{document} 
\begin{frame}{}
   \maketitle 
\end{frame}

\begin{frame}{Outline}
    \begin{itemize}
        \item $\sigma$-algebra; measurable space; measurable map;\\ $\sigma$-algebra generated by x; borel measurable  
        \item Indicator functions  
        \item Caratheodory's extension theorem 
        \item Factorisation lemma  
        \item Filtrations  
        \item Conditional probabilities / independence  
        \item Integration and expectation / conditional expectation  
        \item Radon-nikodym derivative 
        \item Fubini-tonelli theorem 
    \end{itemize}
\end{frame}


\begin{frame}{$\sigma$-algebra}
    \begin{block}{ $\sigma$-algebra }
    a set $\f \in 2^ \omega $ is a $\sigma$-algebra if:
    \begin{itemize}
        \item $\omega  \in  \f $ 
        \item  $a^c  \in \f, \forall a \in \f $ (closed under complement)
        \item  $\cup_i a_i  \in \f, \forall a_i  \in \f  $  (closed under countable union)

    \end{itemize}
    \end{block}
    \textbf{Example}: toss a coin one time:
    \begin{itemize}
        \item $\f = \{\{\emptyset \}, \{H, T\}\}$
        \item $\f=\{\{\emptyset \}, \{H\}, \{T\}, \{H, T\}\}$  
    \end{itemize}

\end{frame}

\begin{frame}{Measurable space}
\end{frame}

\begin{frame}{Measurable map}
    \begin{block}{$\f$-measurable}
        $(\omega, \f) $ is a measurable space, $\X$ is arbitary set and $\g \subseteq 2^\X $
        then $\x: \omega \rightarrow \X$ is a 
        $\f/\g$-measurable map if $\forall A  \in \g, x^{-1}(A)  \in \f$ 
    \end{block}

 $\f/\g$-measurable
\end{frame}

\begin{frame}{Borel $\sigma$-algebra}
\end{frame}

\begin{frame}{Random variable}
\end{frame}



\begin{frame}{Filtration}
     
\end{frame}

\begin{frame}{Independent}
     
\end{frame}

\begin{frame}{Integral}
\end{frame}


\end{document}
