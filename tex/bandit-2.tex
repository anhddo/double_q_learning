\documentclass{beamer} 
\usepackage{amsmath}
\usepackage{amsfonts}
\usetheme{madrid}
\usepackage{graphicx}
\author{Anh Do}
\institute{}
%\date{}
\title{Chapter 2: Foundation of probability}

\newcommand{\f}{\mathcal{F}}
\newcommand{\g}{\mathcal{G}}
\newcommand{\X}{\mathcal{X}}
\newcommand{\x}{\mathbf{x}}
\newcommand{\B}{\mathfrak{B}}
\newcommand{\p}{\mathbb{P}}
\newcommand{\R}{\mathbb{R}}
\newcommand{\E}{\mathbb{E}}
\DeclareMathOperator*{\argmax}{argmax} 
\DeclareMathOperator*{\argmin}{argmin} 

\begin{document} 
\begin{frame}{}
   \maketitle 
\end{frame}

\begin{frame}{Outline}
    \begin{itemize}
        \item $\sigma$-algebra; measurable space; measurable map;\\ 
             Borel  $\sigma$-algebra
        \item Indicator functions  
        \item Caratheodory's extension theorem 
        \item Filtrations  
        \item Conditional probabilities / independence  
        \item Integration and expectation / conditional expectation  
    \end{itemize}
\end{frame}

\begin{frame}{Probability space}
    \begin{itemize}
        \item  The triplet $\left(\Omega, \f, \p  \right)$  is 
            called \textbf{probability space} 
        \item  $\left(\Omega, \f\right)$  is called \textbf{measurable space} 
        \item  $\Omega $ is the domain space
        \item  $\f$  is $\sigma$-algebra 
        \item  $\p$  is the \textbf{measure} 
    \end{itemize}

     
\end{frame}



\begin{frame}{$\sigma$-algebra}
    \begin{block}{ $\sigma$-algebra }
    A set $\f \in 2^ \Omega $ is a $\sigma$-algebra if:
    \begin{itemize}
        \item $\Omega  \in  \f $ 
        \item  $A^c  \in \f, \forall A \in \f $ (closed under complement)
        \item  $\cup_i A_i  \in \f, \forall A_i  \in \f  $  (closed under countable union)
    \end{itemize}
    \end{block}
    \textbf{Example}:Toss a coin one time  $\left(\Omega=\{H, T\} \right)$:
    \begin{itemize}
        \item $\f = \{\{\emptyset \}, \{H, T\}\}$
        \item $\f=\{\{\emptyset \}, \{H\}, \{T\}, \{H, T\}\}$  
    \end{itemize}
    \hrule

    \pause
    \begin{itemize}
        \item A set  $A  \in \f $ is a \textbf{measurable set}.
        \item  $\g$  is a  \textbf{sub-$\sigma$-algebra} if $\f$ is a
            $\sigma$-algebra and $\g \subseteq \f$ 
    \end{itemize}

\end{frame}

\begin{frame}{Probability measure}
    \begin{block}{Probability measure  $\p$ }
        A function  $\p: \f \rightarrow \R $ is a  \textbf{probability measure} if: 
        \begin{itemize}
            \item  $\p \left(\Omega \right) =1$ 
            \item  $\p \left(A\right) > 0 , \forall A  \in \f  $  
            \item  $\p \left(A ^{c}\right) = 1- \p \left(A\right), \forall 
                A  \in \f $  
            \item  $\p \left(\cup _{i }A _{i }\right) 
                = \sum _{i } \p \left(A _{i }\right) $ where $\{A _{i }\}_i $ 
                is a disjoint set and $A _{i }  \in \f $ 
        \end{itemize}
    \end{block}

    \includegraphics[width=.6\textwidth]{img/File_001}
\end{frame}


\begin{frame}{Measurable map}
    \begin{block}{$\f/\g$-measurable map}
        $(\Omega, \f) $ is a measurable space, $\X$ is arbitary set and $\g \subseteq 2^\X $
        then $X: \Omega \rightarrow \X$ is a 
        $\f/\g$-measurable map if $\forall A  \in \g, X^{-1}(A)  \in \f$ 
    \end{block}
    \begin{itemize}
        \item  $X^{-1}(A)  \in \f$: preimage is in $\f$ 
        \item  $\g$  does not need to be a $\sigma$-algebra 
    \end{itemize}

    \pause

    \begin{block}{Smallest  $\sigma$-algebra}
        Let $\mathcal {A}$ is a set of all $\sigma$-algebra that contain $\g$, 
        define  
        $\sigma \left(\g \right) $ is a smallest  $\sigma$-algebra that 
        contains $\g$, where:
        \[
            \f ^{*}= \bigcap _{\f  \in \mathcal {A } } \f   
        \]
        When $\g$ is a set of all open intervals then 
        $\sigma (\g) = \B \left(\R\right)$ is called Borel 
        $\sigma$-algebra of $\R$ 
    \end{block}
\end{frame}

\begin{frame}{Random variable}
    \begin{block}{Random variable}
        A random variable on a measurable space  
        $\left(\Omega, \f \right)$ is  $\f/ \B(\R)$-measurable function  
        $X: \Omega \rightarrow \R  $ 
    \end{block}
    \includegraphics[width=.6 \textwidth]{img/rv}


\end{frame}

\begin{frame}{Indicator functions}
    \includegraphics[width=1\textwidth]{img/indicator}
     
\end{frame}


\begin{frame}{Filtration}
if $\left(\Omega_{1}, \mathcal{F}_{1}\right), \ldots,\left(\Omega_{n}, \mathcal{F}_{n}\right)$ 
    are measureable spaces, then the cartesian product of 
    $\mathcal{F}_{1}, \ldots \mathcal{F}_{n}$ is 
$\mathcal{F}_{1} \times \cdots \times \mathcal{F}_{n}=\left\{A_{1} \times \cdots \times A_{n}: A_{1} \in \mathcal{F}_{1}, \ldots, A_{n} \in \mathcal{F}_{n}\right\} \subseteq 2^{\Omega_{1} \times \cdots \times \Omega_{n}}$


\end{frame}

\begin{frame}{Filtration}
    \includegraphics[width=1 \textwidth]{img/theorem1}
\end{frame}



\begin{frame}{Filtration}
We imagine a learner is sequentially observing
the values of these random variables. First  $X_1$, then  $X_2$ and so on.
Let $X_{1: t} \doteq\left(X_{1}, \ldots, X_{t}\right)$ is the sequentially
observed variables.
    \begin{itemize}
        \item  $\sigma$-algebra $\mathcal{F}_{t}=\sigma\left(X_{1: t}\right)$
        \item   $\mathcal{F}_{0}=\{\emptyset, \Omega\}$
        \item  $\mathcal{F}_{0} \subseteq \mathcal{F}_{1} \subseteq \mathcal{F}_{2} \subseteq \cdots \subseteq \mathcal{F}_{n} \subseteq \mathcal{F}$  
            which corresponding to the  \textbf{increasing knowledge} 
    \end{itemize}

\end{frame}

\begin{frame}{Filtration}
    \begin{block}{Filtration}
        Given a measurable space $\left(\Omega, \f  \right)$, a filtration is
        a sequence $(\f_t) _{t=0} ^{n}$ of a sub $\sigma$-algebra of $\f$ where
        $\f_t \subseteq \f _{t+1}, \forall t<b$.
        \[
            \f _{n}= \sigma \left(\bigcup _{t=0} ^{n} \f_t\right)    
        \]
    \end{block}
\end{frame}

\begin{frame}{Example: Toss a coin 2 times}
    \includegraphics[width=0.7\textwidth]{img/filtration}
\end{frame}



\begin{frame}{Conditional Probabilities}
    \begin{block}{Conditional Probabilities}
    Let $(\Omega, \mathcal{F}, \mathbb{P})$ is a probability space, and let
    $A, B  \in \f $ such that $\p (B) > 0$.
The conditional probability $\mathbb{P}(A \mid B)$ of A given B is defined
as
        \[
     \mathbb{P}(A \mid B)=\frac{\mathbb{P}(A \cap B)}{\mathbb{P}(B)}
        \]
    \end{block}
    \begin{itemize}
        \item $\mathbb{P}(A \mid B)$ is also called the a  \textbf{posteriori} ('after the fact')
        \item The a  \textbf{priori} probability is $\mathbb{P}(A)$
    \end{itemize}
\end{frame}

\begin{frame}{Independence}
    \begin{block}{Independent event}

    Given a probability space $(\Omega, \mathcal{F}, \mathbb{P})$. 
    Two events $A, B  \in \f $ are  \textbf{independent} if:
    \[
      \mathbb{P}(A \cap B)=\mathbb{P}(A) \mathbb{P}(B)   
    \]
    \end{block}

    \begin{block}{Independent r.v}
        \begin{itemize}
            \item  Two collections of events  $\g_1, \g_2$ are said to be independent of
each other if for any  $A  \in \g_1, B  \in \g_2  $ it holds that A and B are independent.
\item  
Two random variables X and Y are independent 
                if  $\sigma(X) $ and  $\sigma(Y) $ are independent of each
other. 
        \end{itemize}


    \end{block}
\end{frame}
\begin{frame}{Example: Roll a fair dice}
    \includegraphics[width=.8\textwidth]{img/independence}
\end{frame}



\begin{frame}{Integration and expectation}
    \begin{block}{Expectation}
    Let $(\Omega, \mathcal{F}, \mathbb{P})$ is a probability space, and random
    variable $X: \Omega \rightarrow \R$ the expectation is denoted by 
    $\E\left[X\right]$ :
    \[
        \mathbb{E}[X]=\int_{\Omega} X(\omega) \mathrm{d} \mathbb{P}(\omega)
    \]
    In particular, for any real-valued, $f: \X \rightarrow \R $ measurable function:
    \[
      \mathbb{E}[f(X)]=\int_{\mathcal{X}} f(x) \mathrm{d} \mathbb{P}_{X}(x)
    \]
    \end{block}

\end{frame}

\begin{frame}{Conditional expectation}
    \includegraphics[width=1\textwidth]{img/cond_exp}
\end{frame}

\begin{frame}{Useful properties}
    \includegraphics[width=1\textwidth]{img/property}
\end{frame}

\end{document}
